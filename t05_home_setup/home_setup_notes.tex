\chapter{Home setup [ASYNC]}
\label{ch:home-setup}

\textbf{Please work through this before Thurs 05-OCT.}

These instructions assume a Windows OS.
You're welcome to try Mac / Linux. 

\section{PuTTY}

We will use the PuTTY terminal emulator and SSH client.
Please install it from:\\
\url{https://www.chiark.greenend.org.uk/~sgtatham/putty/latest.html}

When PuTTY is installed you should see it in the start menu.

\section{Git for Windows}

We will use Git for working with code as we go along.
Please install it from: \\
\url{https://github.com/git-for-windows/git/releases/download/v2.42.0.windows.2/Git-2.42.0.2-64-bit.exe}

When installing be sure to select the option that includes Git in the system path.

When Git is installed correctly you should be able to run Git from PowerShell and see something like: 
\begin{verbatim}
usage: git [--version] [--help] [-C <path>] [-c <name>=<value>]
           [--exec-path[=<path>]] [--html-path] [--man-path] [--info-path]
           [-p | --paginate | -P | --no-pager] [--no-replace-objects] [--bare]
           [--git-dir=<path>] [--work-tree=<path>] [--namespace=<name>]
           [--super-prefix=<path>] [--config-env=<name>=<envvar>]
           <command> [<args>]

...
\end{verbatim}

\section{AWS CLI}

We'll use the AWS CLI later on to work with AWS services directly via their API from the command line.
Please install it from:\\
\url{https://awscli.amazonaws.com/AWSCLIV2.msi}

When AWS is installed you should be able to type \texttt{aws} at the PowerShell prompt and see something like:
\begin{verbatim}
usage: aws [options] <command> <subcommand> [<subcommand> ...] [parameters]
To see help text, you can run:

  aws help
  aws <command> help
  aws <command> <subcommand> help
aws: error: the following arguments are required: command
\end{verbatim}

\section{Other software}

We may need other software and will install it as we go along.

