\chapter{EC2 Windows Lab [ASYNC]}
\label{ch:ec2-windows-lab-async}

\section{Video tutorials}

\subsection{Demonstration}
\url{https://media.heanet.ie/page/76cad179a16a4a0d9f4004d619d1eb85}

\subsection{Lab walk-through}
\url{https://media.heanet.ie/page/f56227a4bf8e485388247272efc630f7}

\section{Independent work}

For this week's independent work exercise please work through the creation of an EC2 Windows machine from scratch.
This is as much to build familiarity with the VPC ecosystem as with EC2 or Windows Server itself.

\subsection{Optional: SSH on Windows}

Windows Server can easily provide SSH capability to its Command-Line Interface.
\begin{enumerate}
\item Work through the instructions from \autoref{sec:ssh-server-on-windows} onwards to enable it.
\item You will need to add SSH (TCP Port 22) to the security group.
\item Just put the public IP into PuTTY. Username is Administrator and password is the windows password.
\end{enumerate}
You'll end up at the older \texttt{cmd.exe} prompt.
Just type \texttt{powershell} and press enter to get PowerShell instead.

\subsection{Optional: 2 instances in a subnet}

You could try to add a second EC2 instance to your VPC \& subnet that runs linux.
You will need:
\begin{enumerate}
\item An ED25519 key pair
\item To add SSH (TCP Port 22) to your Inbound rules of your security group. 
\end{enumerate}

