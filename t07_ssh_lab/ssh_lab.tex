\chapter{SSH lab [Async]}
\label{ch:ssh-lab-async}

\section{Video tutorial}

\url{https://media.heanet.ie/page/9132bc09bc8f4195b44f602e914bb1c0}

\section{Exercise}

\begin{enumerate}

\item
  Use PuTTYGen to create an ED25519 SSH key.

\item
  Save the private key in ppk format.

\item
  Import the key to your AWS account (copying the Public Key for Pasting into authorized_keys).

\item
  Construct the same setup as in \autoref{ch:ec2-lab-async} except that you should choose the Key in the dropdown when creating the EC2 instance.

\item
  Use PuTTY to connect to your instance:
  \begin{enumerate}
  \item Public IPv4 address goes into the hostname / IP address box.
  \item Connection / Data / Auto-Login username should be \texttt{ec2-user}.
  \item Connection / SSH / Auth Private Key should be your private key file.
  \item Consider saving your connection data to avoid having to re-do it.
  \end{enumerate}

\item
  Hit Connect.
  First time you connect to any machine it'll give a warning message.
  You can just hit continue / accept.

\item 
  You'll be then asked your Key Passphrase.
  If this succeeds you'll see the Bash prompt.

\end{enumerate}

In later labs we'll do more work at the bash prompt.
For now, you've set up your first linux instance with SSH access.

