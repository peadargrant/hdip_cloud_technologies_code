\chapter{AWS setup}
\label{ch:aws-setup}

The labs in this course will use Amazon Web Services, or AWS.
To do the labs in this course, you will need your own AWS account.

Note that this is not a philosophical endorsement of Amazon!
There are other cloud providers too - IBM, Google, Microsoft Azure, others.
Most of the concepts encountered in AWS translate to the others.

\section{Charges}
\label{sec:charges}

Almost all AWS services are chargeable.
Many services have a time-limited free tier.
We will stay almost entirely within the free tier. 

You will need a credit / debit card to sign up for AWS.
If you don't have one you should be able to use a prepaid card or use an online card like Revolut.
\textit{I have not tested this option.}

\section{Sign up for an AWS account}
\label{sec:signup}

If you already have an AWS account you should skip this section.

Sign up for an AWS account by visiting the link:\\
\url{https://portal.aws.amazon.com/billing/signup}

There is a 5-step process to signing up.

For the account type you should choose Personal.

\section{Log in to the console}

Make sure that you can login to the AWS console using the credentials from the \autoref{sec:signup}.
The AWS console can be accessed at:\\
\url{https://aws.amazon.com/console/}

Bookmark the AWS console link in your Browser. You will need it often.


\subsection{Region setup}
\label{sec:region-setup}

AWS is divided into a number of regions (next class!).

On the top right of the AWS Console, it will probably say \textit{N Virginia}.
Click this and change it to Ireland (eu-west-1).
The college firewall only allows some connections we need for our work in this module to AWS resources in the eu-west-1 region.

We will look at regions again as part of our study of Global Infrastructure.

\section{Billing alarm}

As stated in \autoref{sec:charges} we will aim to stay within the free tier where possible.
To avoid any unexpected charges, you will set up a Billing alarm on your account.
This uses a service named CloudWatch.
You will also familiarise yourself with using AWS documentation to help you work with AWS.
The lab instructions \textit{will NOT} provide step-by-step guides!

Visit the \href{https://docs.aws.amazon.com/AmazonCloudWatch/latest/monitoring/monitor_estimated_charges_with_cloudwatch.html}{documentation for CloudWatch Billing alarms}.

\subsection{Enabling billing alerts}

Follow the 4 steps under ``To enable the monitoring of estimated charges''.

You should also turn on the ``Receive Free Tier Usage Alerts'' tickbox.

\subsection{Creating a billing alarm}

Follow instructions 1 to 10 (of the 13) under ``To create a billing alarm using the CloudWatch console''.
Recommended alarm level is 10 US dollars.

When you have done the 10 steps, click ``Create New Topic''.
A default Topic name will be created.
Press Create Topic.

Fill in your e-mail address in the box.
Then press Next to go on to the next screen.

Set the name to \texttt{BILLING\_ALARM} and continue to the next step. 

Then press Create Alarm.
You should see a green bar with a message like \texttt{Successfully created alarm BILLING\_ALARM}.

Check your e-mail and confirm the subscription.

\subsection{Checking your billing alarm}

To confirm that your billing alarm has been set up, you will run a provided script.
\begin{enumerate}
\item Click the CloudShell icon \texttt{(\textrangle\_)}.
  This will open a command shell, which will take some time to initialise.

\item Wait until you see something like:
\begin{verbatim}
[cloudshell-user@ip-10-0-187-141 ~]$ 
\end{verbatim}

\item Click the Actions button and then Upload File.

\item Select the file \texttt{check\_billing\_alarm.py} and hit Upload.
  Wait for the upload to complete.

\item Then type:
\begin{verbatim}
python3 ./check_billing_alarm.py
\end{verbatim}

\item Confirm that your billing alarm has been set up from the script.
  If not, fix it and re-run the script to confirm. 

\end{enumerate}

