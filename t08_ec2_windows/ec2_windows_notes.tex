\chapter{EC2 Windows}
\label{ch:ec2-windows}

\section{Remote access}

Windows Server is traditionally managed using its GUI, although remote
PowerShell use through various means is becoming more popular.

GUI management can be done via a directly or KVM-connected video
display, mouse and keyboard. As with linux instances, AWS doesn't
provide emulated KVM access. Only a screenshot picture of the instance
can be taken.

\subsection{Remote desktop protocol}\label{remote-desktop-protocol}

Remote Desktop Protocol
(RDP) is the standard way to manage remote windows servers (and clients)
and is built into Microsoft Windows. Key things to note:

\begin{itemize}
\item
  RDP runs over TCP on port 3389.
\item
  RDP must be enabled on the Windows control panel of the server for it
  to work. AMIs for Windows on AWS have RDP enabled already.
\item
  RDP sessions are not as ``independent'' of each other as SSH sessions
  are:

  \begin{itemize}
  \item
    Generally Windows Server supports a maximum of two concurrent
    separate user logins over a combination of RDP (and the local
    console, which isn't present on AWS).
  \item
    Windows Clients (e.g.~XP, 7, 8, 10) normally support only one login
    at a time (over all methods). If the same user logs in, the existing
    session should be picked-up/transferred.
  \end{itemize}
\item
  RDP allows a number of different enhancements to be selected when
  connecting:

  \begin{itemize}
  \item
    The resolution of the remote session can be set when connecting.
  \item
    Current RDP versions allow multi-monitor support.
  \item
    File sharing can be enabled for the session so that necessary files
    can be easily transferred between the client and RDP server.
  \item
    Ports can also be attached.
  \item
    Audio and printer output can be redirected back to the client.
  \end{itemize}
\item
  RDP does not support user managed keys like SSH, but can support
  single sign-on through Active Directory if both the client and server
  share the same domain.
\end{itemize}

\subsection{Terminal Services}\label{terminal-services}

Terminal Services is a licensing extension to remote desktop to enable a
number of different concurrent remote sessions to be open. It is
sometimes used to allow users to ``pick up'' their desktop at different
computers, often thin clients.

\subsection{Alternatives}\label{alternatives}

\begin{description}
\item[VNC]
is often used in Windows to present the console remotely, in either
view-only or view \& control mode. VNC is cross platform and has clients
for almost every operating system, but is limited somewhat by its weak
authentication.
\item[Proprietary solutions]
like TeamViewer and GoToMyPC require additional client software, can be
easier to setup for non-technical users and often can traverse NAT
routers automatically.
\end{description}

\subsection{Command-line access}\label{command-line-access}

\begin{itemize}
\item
  \textbf{PowerShell remoting} gives access to the PowerShell interface
  of a remote Windows computer. It can be used on a basic level like
  SSH, but underneath is quite different, with different strengths and
  weaknesses.
\item
  \textbf{SSH} access to Windows is possible using OpenSSH, offering
  access to \texttt{cmd.exe} or \texttt{PowerShell} from remote
  computers.
\end{itemize}

% \section{VPC setup}\label{vpc-setup}

% VPC for this week in the file \texttt{ec2\_windows\_vpc\_setup.ps1}

% This is largely the same as the linux Lab except that:

% \begin{itemize}
% \item
%   Checks for the existence of your \texttt{LAB\_KEY} before continuing.
% \item
%   Creates the security group \texttt{LAB\_SG}
% \item
%   Adds a rule to \texttt{LAB\_SG} to permit RDP (Port 3389) access.
% \end{itemize}

% Output should be similar to:

% \begin{verbatim}
% KeyPair: key-09a3afe83db8e8381
% VPC: vpc-0209bd43322117a41
% Subnet: subnet-05fa961d022b72730
% IGW: igw-0d070969dfd43c2cf
% Route Table: rtb-0b56038dfec860e3d
% {
%     "Return": true
% }
% Security Group: sg-0326166674b0b68f0
% Modified security group to permit RDP access
% \end{verbatim}

% \section{Lookup Image Ids automatically (move for
% 2021)}\label{lookup-image-ids-automatically-move-for-2021}

% Image IDs are region and account-dependent. They also get updated as
% Amazon update the images.

% \begin{verbatim}
% # print out list of Windows AMIs
% ami ssm get-parameters-by-path -path /aws/service/ami-windows-latest --query "Parameters[].Name" 
% \end{verbatim}

% The ``standard'' windows image we will use is
% \texttt{Windows\_Server-2019-English-Full-Base}.

\section{Provisioning a Windows Instance}
\label{sec:provisioning-a-windows-instance}

These steps assume you are familiar with provisioning a linux instance
from the previous week's lab.

\subsection{Launching the instance}
\label{sec:launching-the-instance}

To start a Windows instances, we mainly need the same set of
information:

\begin{itemize}
\item
  The AMI of the Windows image. We will use \emph{Microsoft Windows
  Server 2022 Base}. Check the AMI as it appears on your own account.
\item
  The Subnet ID to launch the instance into. (Which links us to a specific VPC)
\item
  The instance type 
\item
  Security Group
\item
  The \textbf{SSH key pair}:
\end{itemize}

The Key Pair is still needed, but it serves a different function in
Windows instances compared to Linux instances.


\section{Connecting over RDP}\label{connecting-over-rdp}

To connect to our Windows instance over RDP we need to know the Public
IP and the password for the administrator account.

\subsection{Getting password data}\label{getting-password-data}

To get the password data for a given instance ID in the AWS Console, select the Instances.
Under Actions / Security select Get Windows Password.
Browse for your Private Key file.

If the password is blank, it means that the post-launch setup hasn't yet
completed. Just pause here and try it again in a couple of minutes.

The decrypted password shown matches the Windows local Administrator
account.

\subsection{Connecting with the graphical client}
\label{sec:connecting-with-the-graphical-client}

Look for Remote Desktop Connnection in the start menu. Use the IP
address, username Administrator, and the password and hit Connect.

You should be at a standard Windows desktop running on the remote
machine. At this point you can administer the machine any way you want,
including setting up a new admin user account and/or changing the
password.

\section{SSH server on Windows}
\label{sec:ssh-server-on-windows}

Remote Desktop is just one way to access a Windows server for management.
Windows now supports SSH access to the command-prompt (and to PowerShell).

\subsection{Checking if OpenSSH is installed}
\label{checking-if-openssh-is-installed}

On your AWS Windows Server open PowerShell and type

\begin{verbatim}
Get-WindowsCapability -Online -Name Open*
\end{verbatim}

You will see something like:

\begin{verbatim}
Name         : OpenSSH.Client~~~~0.0.1.0
State        : Installed
DisplayName  : OpenSSH Client
Description  : OpenSSH-based secure shell (SSH) client, for secure key management and access to remote machines.
DownloadSize : 1323493
InstallSize  : 5301402

Name         : OpenSSH.Server~~~~0.0.1.0
State        : NotPresent
DisplayName  : OpenSSH Server
Description  : OpenSSH-based secure shell (SSH) server, for secure key management and access from remote machines.
DownloadSize : 1297677
InstallSize  : 4946932
\end{verbatim}

To install the OpenSSH server we type

\begin{verbatim}
Add-WindowsCapability -Online -Name OpenSSH.Server~~~~0.0.1.0
\end{verbatim}

If it succeeded you'll see

\begin{verbatim}
Path          :
Online        : True
RestartNeeded : False
\end{verbatim}

Re-run the \texttt{Get-WindowsCapability} command to confirm it's now
installed.

\subsection{Server startup}\label{server-startup}

SSH is provided by a server process called \texttt{sshd}.

First we confirm the service exists:

\begin{verbatim}
Get-Service sshd
\end{verbatim}

From the output we note that it is installed but isn't currently
running:

\begin{verbatim}
Status   Name               DisplayName
------   ----               -----------
Running  sshd               OpenSSH SSH Server
\end{verbatim}

This server needs to be started and to be configured to start
automatically on reboot:

\begin{verbatim}
# start immediately
Start-Service sshd

# Set to start automatically on boot
Set-Service -Name sshd -StartupType automatic
\end{verbatim}

\subsection{Testing SSH server from same EC2
instance}\label{testing-ssh-server-from-same-ec2-instance}

Now we can test it by connecting from the EC2 instance itself.

\begin{verbatim}
ssh Administrator@localhost
\end{verbatim}

and entering the admin password. We're actually at the older command.com
prompt, but this can be changed to Powershell using system
configuration.

\subsection{Connecting from our computer to EC2 SSH
server}\label{connecting-from-our-computer-to-ec2-ssh-server}

The main reason to use SSH would be to connect from our own local
computer to the EC2 instance. SSH uses TCP port 22, which isn't
currently permitted by the security group. There are two possibilities:
create a second security group and assign it to the machine, or modify
the existing group.


Then on our local computer we type:

\begin{verbatim}
ssh Administrator@$PublicIp
\end{verbatim}

and put in the machine's password. (Notice here that although AWS did
setup the private keys for us, that it doesn't inject the private key
into the EC2 instance for us as with Linux.)

You have successfully connected over SSH if you see output similar to
the following:

\begin{verbatim}
Microsoft Windows [Version 10.0.17763.1518]
(c) 2018 Microsoft Corporation. All rights reserved.

administrator@EC2AMAZ-3DE14BE C:\Users\Administrator>
\end{verbatim}

This is actually the older command.com shell, rather than PowerShell. We
can run powershell from here by just typing:

\begin{verbatim}
powershell
\end{verbatim}

Typing \texttt{exit} will get us back out of Powershell, and
\texttt{exit} again will close the SSH session.

\subsection{Configuring SSH}\label{configuring-ssh}

If you intend to use SSH to work with Windows regularly you should take
the time to configure it properly. Here are a few tips:

\begin{itemize}
\item
  There are a number of ways to reconfigure SSHd to launch PowerShell
  instead, with various drawbacks.
\item
  Use the local user/group management to limit who is allowed to login
  over SSH.
\item
  Use public/private keys and disable password authentication. You will
  have to consult the OpenSSH docs for Windows as the configuration is
  slightly different to Linux.
\item
  If using SSH keys, learn about agent forwarding.
\end{itemize}

